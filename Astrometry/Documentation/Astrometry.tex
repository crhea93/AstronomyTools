%%% Research Diary - Entry
%%% Template by Mikhail Klassen, April 2013
%%% 
\documentclass[11pt,letterpaper]{article}
\newcommand{\workingDate}{\textsc{2018 $|$ June}}
\newcommand{\userName}{Carter Rhea}
\newcommand{\institution}{Universite de Montreal}
\usepackage{python}

\usepackage[]{algorithm2e}

\usepackage{listings}
\usepackage{color}
\definecolor{dkgreen}{rgb}{0,0.6,0}
\definecolor{gray}{rgb}{0.5,0.5,0.5}
\definecolor{mauve}{rgb}{0.58,0,0.82}
\usepackage{hyperref}
\hypersetup{
	colorlinks=true,
	linkcolor=blue,
	filecolor=magenta,      
	urlcolor=cyan,
}
\lstset{frame=tb,
	language=Java,
	aboveskip=3mm,
	belowskip=3mm,
	showstringspaces=false,
	columns=flexible,
	basicstyle={\small\ttfamily},
	numbers=none,
	numberstyle=\tiny\color{gray},
	keywordstyle=\color{blue},
	commentstyle=\color{dkgreen},
	stringstyle=\color{mauve},
	breaklines=true,
	breakatwhitespace=true,
	tabsize=3
}
\usepackage{researchdiary_png}
% To add your univeristy logo to the upper right, simply
% upload a file named "logo.png" using the files menu above.

\begin{document}
	\univlogo
	
	\title{Documentation for WVT}
	
	%\begin{python}%
	%print r"Hello \LaTeX!"
	%\end{python}%
	\textit{Documentation for Astrometry Routines}
	
	\tableofcontents
	
	\newpage
	
	
	\newpage
\section{LScalc.py}
This program has a very simple goal: calculate the linear distance of an astronomical object given a redshift and angular view. To do this, I simply utilized the following set of equations to calculate the quantity of intereset, $l$.
\begin{center}
	\begin{tabular}{|c|c|}
		\hline 
		Parameter & Brief Description \\ 
		\hline 
		$l$ & Linear Distance in Kiloparsecs \\ 
		\hline 
		$z$ & Redshift \\
		\hline
		$d_A(z)$ & Angular Distance \\
		\hline 
		$\theta$ & Angular Seperation in arcseconds \\
		\hline
		$d_H$ & Hubble Distance \\
		\hline
		$c$ & Speed of Light \\
		\hline
		$H_0$ & Hubble Constant \\
		\hline
		$d_C$ & Comoving Distance \\
		\hline 
		$E(z)$ & Energy Function \\
		\hline 
		$d_M(z)$ & Moving Distance \\
		\hline 
		$\Omega_{rel}$ & Mass Density of Relativistic Particles \\
		\hline
		$\Omega_{mass}$ & Mass Density of Baryonic and NonBaryonic Particles \\
		\hline
		$\Omega_{\Lambda}$ & Mass Density of Dark Energy \\
		\hline
	\end{tabular} 
\end{center}



\begin{equation}
l = d_A(z)*\theta
\end{equation}

\begin{equation}
d_A(z) = \frac{d_M(z)}{1+z}
\end{equation}	

\begin{equation}
	d_M(z) = \left\{ \begin{array}{ll}
	\frac{d_H}{\sqrt{\Omega_k}}\sinh\Big({\sqrt{\Omega_k}\frac{d_C(z)}{d_H}}\Big) & \Omega_k > 0 \\
	d_C(z) & \Omega_k = 0 \\
	\frac{d_H}{\sqrt{|\Omega_k|}}\sin\Big({\sqrt{|\Omega_k|}\frac{d_C(z)}{d_H}}\Big) & \Omega_k < 0 \\
	\end{array} \right.	 
\end{equation}	

\begin{equation}
	d_C(z) = d_H\int_{0}^{z} \frac{dz'}{E(z')}
\end{equation}

\begin{equation}
	E(z) = \sqrt{\Omega_{rel}(1+z)^4+\Omega_{mass}(1+z)^3+\Omega_k(1+z)^2+\Omega_{\Lambda}}
\end{equation}

\begin{equation}
	\Omega_k = 1-\Omega_{mass}-\Omega_{\Lambda} 
\end{equation}

\begin{equation}
	d_H = \frac{c}{H_0}
\end{equation}

\section{AScalc.py}
This program is almost identitical to \textbf{LScalc.py} except that instead of solving for the linear distance we are solving for the angular seperation.
\begin{center}
	\begin{tabular}{|c|c|}
		\hline 
		Parameter & Brief Description \\ 
		\hline 
		$l$ & Linear Distance in Kiloparsecs \\ 
		\hline 
		$z$ & Redshift \\
		\hline
		$d_A(z)$ & Angular Distance \\
		\hline 
		$\theta$ & Angular Seperation in arcseconds \\
		\hline
		$d_H$ & Hubble Distance \\
		\hline
		$c$ & Speed of Light \\
		\hline
		$H_0$ & Hubble Constant \\
		\hline
		$d_C$ & Comoving Distance \\
		\hline 
		$E(z)$ & Energy Function \\
		\hline 
		$d_M(z)$ & Moving Distance \\
		\hline 
		$\Omega_{rel}$ & Mass Density of Relativistic Particles \\
		\hline
		$\Omega_{mass}$ & Mass Density of Baryonic and NonBaryonic Particles \\
		\hline
		$\Omega_{\Lambda}$ & Mass Density of Dark Energy \\
		\hline
	\end{tabular} 
\end{center}
\begin{equation}
	\theta = \frac{l}{d_A(z)}
\end{equation}

\begin{equation}
	d_A(z) = \frac{d_M(z)}{1+z}
\end{equation}

\begin{equation}
d_M(z) = \left\{ \begin{array}{ll}
\frac{d_H}{\sqrt{\Omega_k}}\sinh\Big({\sqrt{\Omega_k}\frac{d_C(z)}{d_H}}\Big) & \Omega_k > 0 \\
d_C(z) & \Omega_k = 0 \\
\frac{d_H}{\sqrt{|\Omega_k|}}\sin\Big({\sqrt{|\Omega_k|}\frac{d_C(z)}{d_H}}\Big) & \Omega_k < 0 \\
\end{array} \right.	 
\end{equation}	

\begin{equation}
d_C(z) = d_H\int_{0}^{z} \frac{dz'}{E(z')}
\end{equation}

\begin{equation}
E(z) = \sqrt{\Omega_{rel}(1+z)^4+\Omega_{mass}(1+z)^3+\Omega_k(1+z)^2+\Omega_{\Lambda}}
\end{equation}

\begin{equation}
\Omega_k = 1-\Omega_{mass}-\Omega_{\Lambda} 
\end{equation}

\begin{equation}
d_H = \frac{c}{H_0}
\end{equation}
\end{document}

