%%% Research Diary - Entry
%%% Template by Mikhail Klassen, April 2013
%%% 
\documentclass[11pt,letterpaper]{article}
\newcommand{\workingDate}{\textsc{2018}}
\newcommand{\userName}{Carter Rhea}
\newcommand{\institution}{Universite de Montreal}
\usepackage{python}

\usepackage[]{algorithm2e}

\usepackage{listings}
\usepackage{color}
\definecolor{dkgreen}{rgb}{0,0.6,0}
\definecolor{gray}{rgb}{0.5,0.5,0.5}
\definecolor{mauve}{rgb}{0.58,0,0.82}
\usepackage{hyperref}
\hypersetup{
	colorlinks=true,
	linkcolor=blue,
	filecolor=magenta,      
	urlcolor=cyan,
}
\lstset{frame=tb,
	language=Java,
	aboveskip=3mm,
	belowskip=3mm,
	showstringspaces=false,
	columns=flexible,
	basicstyle={\small\ttfamily},
	numbers=none,
	numberstyle=\tiny\color{gray},
	keywordstyle=\color{blue},
	commentstyle=\color{dkgreen},
	stringstyle=\color{mauve},
	breaklines=true,
	breakatwhitespace=true,
	tabsize=3
}
\usepackage{researchdiary_png}
% To add your univeristy logo to the upper right, simply
% upload a file named "logo.png" using the files menu above.

\begin{document}
	\univlogo
	
	\title{Documentation for PSF Creation}
	
	%\begin{python}%
	%print r"Hello \LaTeX!"
	%\end{python}%
	\textit{Documentation for PSF Creation}
	
	\tableofcontents
	
	\newpage
	
	
	\newpage
\section{PSF.py}
The following python file utilizes the \textbf{ChaRT} form found on the following website:
$$\href{http://cxc.harvard.edu/ciao/PSFs/chart2/runchart.html}{http://cxc.harvard.edu/ciao/PSFs/chart2/runchart.html} $$

Instead of filling in the form manually, we will allow python to do our bidding. Because of this, there are several python modules that must be installed: \texttt{selenium}, \texttt{chromium-chromedrive}\footnote{Also need google chrome installed}, \texttt{tarfile}, and \texttt{wget}. These can all be installed with the basic \texttt{pip install module\_name} command. 

The program will calculate all necessary ChaRT parameters for you if you wish; otherwise you can enter the Effective Monochromatic Energy and Net Energy Flux when prompted by the command terminal. A prompt will be initiated upon running the python file which will ask if you want to calculate the chart parameters. If you do type \texttt{yes}, and if you dont type anything else! 

Since the program requires data from you email server (It's set up for gmail only right now), you need to create a simple txt file containing the following two lines:

$$gmail\_account = account\_name$$
$$gmail\_password = password$$

You will then feed the full path of this file into the python script. Since I share my code on GitHub, I find this to be the safest way to read in private information.

\begin{algorithm}[H]\label{algo:PSF}
	\caption{PSF Creation Algorithm}
	\KwData{Basic Reprocessd Data and ARF File for Region}
	\KwResult{PSF file}
	Step 1: Prepare ChaRT Inputs \;
	Step 2: Fill in ChaRT Form \;
	Step 3: Wait until Email from ChaRT arrives and Read Zip Folder Web-Location \; 
	Step 4: Run \texttt{wget} to download the zip folder and then unzip \;
	Step 5: Use \texttt{Marx} to calculate PSF for region \;
\end{algorithm}

Once the file has been downloaded from ChaRT and unzipped, we must run the \texttt{Marx} program\footnote{\href{http://cxc.harvard.edu/ciao/threads/marx/}{http://cxc.harvard.edu/ciao/threads/marx/}}. All necessary inputs for \texttt{Marx} are gathered insitu, thus requiring no user input.

Clearly, marx must be installed. This should be trivial since \texttt{ciao} is assumed to be installed. Run the following command from the terminal to install \texttt{Marx}:
$$install\_marx$$

This only installs \texttt{Marx}, so we still need to initiate the program before we can run \textit{PSF.py}. I suggest accessing your $.bashrc$ profile (assuming you use \textit{sh}) and adding an alias for \texttt{Marx}.

$$\texttt{alias marx="source  /full\_path\_to/setup\_marx.sh"} $$

After initially setting up \texttt{Marx} you will likely get an notification to do this step anyhow.
\end{document}

