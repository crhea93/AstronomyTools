%%% Research Diary - Entry
%%% Template by Mikhail Klassen, April 2013
%%% 
\documentclass[11pt,letterpaper]{article}
\newcommand{\workingDate}{\textsc{2018}}
\newcommand{\userName}{Carter Rhea}
\newcommand{\institution}{Universite de Montreal}
\usepackage{python}

\usepackage[]{algorithm2e}

\usepackage{listings}
\usepackage{color}
\definecolor{dkgreen}{rgb}{0,0.6,0}
\definecolor{gray}{rgb}{0.5,0.5,0.5}
\definecolor{mauve}{rgb}{0.58,0,0.82}
\usepackage{hyperref}
\hypersetup{
	colorlinks=true,
	linkcolor=blue,
	filecolor=magenta,      
	urlcolor=cyan,
}
\lstset{frame=tb,
	language=Java,
	aboveskip=3mm,
	belowskip=3mm,
	showstringspaces=false,
	columns=flexible,
	basicstyle={\small\ttfamily},
	numbers=none,
	numberstyle=\tiny\color{gray},
	keywordstyle=\color{blue},
	commentstyle=\color{dkgreen},
	stringstyle=\color{mauve},
	breaklines=true,
	breakatwhitespace=true,
	tabsize=3
}
\usepackage{researchdiary_png}
% To add your univeristy logo to the upper right, simply
% upload a file named "logo.png" using the files menu above.

\begin{document}
	\univlogo
	
	\title{Documentation for Surface Brightness}
	
	%\begin{python}%
	%print r"Hello \LaTeX!"
	%\end{python}%
	\textit{Documentation for Surface Brightness Calculations}
	
	\tableofcontents
	
	\newpage
	
	
	\newpage
\section{Introduction}
X-ray astronomy is known for a phenomenon aptly-monikered \textit{photon-starvation}; this is amplified in certain cases when we are looking back at incredibly distant objects. Photon-starvation poses as serious issue when astronomers wish to determine whether or not a galactic cluster is a Cool-Core Cluster or not. With a strong signal-to-noise, one can "simply" use Weighted Voronoi Tesselations to construct an appropriate bin-map and then use \textbf{XSPEC} to develop a temperature map of the region. Unfortunately, for cases in which there is a dissapointingly low signal-to-noise, astronomers must resort to other measures: enter Surface Brightness Concentration Value (SBCV). Using this tool, we can comfortably determine whether or not an object is a CCC even with exceptionally low signal-to-noise. I will not delve into the details regarding the virtues of the SBCV but rather encourage any interested party to read the following papers: \cite{Santos2018} \cite{Santos2018a} \cite{Semler2012}. However, it is necessary to define this parameter:

\begin{equation}
	SBCV = \frac{F(R<40kpc)}{F(R<400kpc)}
\end{equation}
where $F$ is the X-ray Flux and $R$ represents the radius of the annulus.

According to \cite{Semler2012}, we are interested in the following regimes:
\[ \begin{cases} 
\texttt{Non Cool Core} & SBCV < 0.075 \\
\texttt{Moderate Cool Core} & 0.075 < SBCV < 0.155 \\
\texttt{Strong Cool Core} & SBCV > 0.155 
\end{cases}
\] 

\section{Algorithm}
The algorithm itself is not particularly complicated, but there are several crucial steps which makes it worth detailing.

\begin{algorithm}[H]\label{algo:BA}
	\caption{Surface Brightness Concentration Value}
	\KwData{Reprocessed Fits File}
	\KwResult{Surface Brightness Concentration Value with Bounds}
	Step 1: Use Astomety tool (\textit{ASCalc.py}) -- located in the \textit{Astrometry} directory -- to calculate the angular seperation for 40kpc and 400 kpc\;
	Step 2: Use this value (in arcseconds) in \textit{ds9} to generate \textit{.reg} files for 40/400kpc\;
	Step 3: Run \textit{Precursor.py} -- located in the \textit{GeneralUse} directory -- to generate \textit{.arf} files for 40/400kpc\;
	Step 4: Calculate Monochromatic Energy AND Surface Brightness Coefficients using \textit{SurfBright.py} which is located in the \textit{SurfaceBrightness} directory\;
	Step 5: Calculate SBCV and bounds using \textit{CSB.py} which is located in the \textit{SurfaceBrightness} directory\;
\end{algorithm}



\newpage


\bibliographystyle{plain}
\bibliography{ref.bib}

\end{document}

